\documentclass[10pt,a4paper]{article}

\usepackage[spanish,activeacute,es-tabla]{babel}
\usepackage[utf8]{inputenc}
\usepackage{ifthen}
\usepackage{listings}
\usepackage{dsfont}
\usepackage{subcaption}
\usepackage{amsmath}
\usepackage[strict]{changepage}
\usepackage[top=1cm,bottom=2cm,left=1cm,right=1cm]{geometry}%
\usepackage{color}%
\newcommand{\tocarEspacios}{%
	\addtolength{\leftskip}{3em}%
	\setlength{\parindent}{0em}%
}

% Especificacion de procs

\newcommand{\In}{\textsf{in }}
\newcommand{\Out}{\textsf{out }}
\newcommand{\Inout}{\textsf{inout }}

\newcommand{\encabezadoDeProc}[4]{%
	% Ponemos la palabrita problema en tt
	%  \noindent%
	{\normalfont\bfseries\ttfamily proc}%
	% Ponemos el nombre del problema
	\ %
	{\normalfont\ttfamily #2}%
	\
	% Ponemos los parametros
	(#3)%
	\ifthenelse{\equal{#4}{}}{}{%
		% Por ultimo, va el tipo del resultado
		\ : #4}
}

\newenvironment{proc}[4][res]{%
	
	% El parametro 1 (opcional) es el nombre del resultado
	% El parametro 2 es el nombre del problema
	% El parametro 3 son los parametros
	% El parametro 4 es el tipo del resultado
	% Preambulo del ambiente problema
	% Tenemos que definir los comandos requiere, asegura, modifica y aux
	\newcommand{\requiere}[2][]{%
		{\normalfont\bfseries\ttfamily requiere}%
		\ifthenelse{\equal{##1}{}}{}{\ {\normalfont\ttfamily ##1} :}\ %
		\{\ensuremath{##2}\}%
		{\normalfont\bfseries\,\par}%
	}
	\newcommand{\asegura}[2][]{%
		{\normalfont\bfseries\ttfamily asegura}%
		\ifthenelse{\equal{##1}{}}{}{\ {\normalfont\ttfamily ##1} :}\
		\{\ensuremath{##2}\}%
		{\normalfont\bfseries\,\par}%
	}
	\renewcommand{\aux}[4]{%
		{\normalfont\bfseries\ttfamily aux\ }%
		{\normalfont\ttfamily ##1}%
		\ifthenelse{\equal{##2}{}}{}{\ (##2)}\ : ##3\, = \ensuremath{##4}%
		{\normalfont\bfseries\,;\par}%
	}
	\renewcommand{\pred}[3]{%
		{\normalfont\bfseries\ttfamily pred }%
		{\normalfont\ttfamily ##1}%
		\ifthenelse{\equal{##2}{}}{}{\ (##2) }%
		\{%
		\begin{adjustwidth}{+5em}{}
			\ensuremath{##3}
		\end{adjustwidth}
		\}%
		{\normalfont\bfseries\,\par}%
	}
	
	\newcommand{\res}{#1}
	\vspace{1ex}
	\noindent
	\encabezadoDeProc{#1}{#2}{#3}{#4}
	% Abrimos la llave
	\par%
	\tocarEspacios
}
{
	% Cerramos la llave
	\vspace{1ex}
}

\newcommand{\aux}[4]{%
	{\normalfont\bfseries\ttfamily\noindent aux\ }%
	{\normalfont\ttfamily #1}%
	\ifthenelse{\equal{#2}{}}{}{\ (#2)}\ : #3\, = \ensuremath{#4}%
	{\normalfont\bfseries\,;\par}%
}

\newcommand{\pred}[3]{%
	{\normalfont\bfseries\ttfamily\noindent pred }%
	{\normalfont\ttfamily #1}%
	\ifthenelse{\equal{#2}{}}{}{\ (#2) }%
	\{%
	\begin{adjustwidth}{+2em}{}
		\ensuremath{#3}
	\end{adjustwidth}
	\}%
	{\normalfont\bfseries\,\par}%
}

% Tipos

\newcommand{\nat}{\ensuremath{\mathds{N}}}
\newcommand{\ent}{\ensuremath{\mathds{Z}}}
\newcommand{\float}{\ensuremath{\mathds{R}}}
\newcommand{\bool}{\ensuremath{\mathsf{Bool}}}
\newcommand{\cha}{\ensuremath{\mathsf{Char}}}
\newcommand{\str}{\ensuremath{\mathsf{String}}}

% Logica

\newcommand{\True}{\ensuremath{\mathrm{true}}}
\newcommand{\False}{\ensuremath{\mathrm{false}}}
\newcommand{\Then}{\ensuremath{\rightarrow}}
\newcommand{\Iff}{\ensuremath{\leftrightarrow}}
\newcommand{\implica}{\ensuremath{\longrightarrow}}
\newcommand{\IfThenElse}[3]{\ensuremath{\mathsf{if}\ #1\ \mathsf{then}\ #2\ \mathsf{else}\ #3\ \mathsf{fi}}}
\newcommand{\yLuego}{\land _L}
\newcommand{\oLuego}{\lor _L}
\newcommand{\implicaLuego}{\implica _L}

\newcommand{\cuantificador}[5]{%
	\ensuremath{(#2 #3: #4)\ (%
		\ifthenelse{\equal{#1}{unalinea}}{
			#5
		}{
			$ % exiting math mode
			\begin{adjustwidth}{+2em}{}
				$#5$%
			\end{adjustwidth}%
			$ % entering math mode
		}
		)}
}

\newcommand{\existe}[4][]{%
	\cuantificador{#1}{\exists}{#2}{#3}{#4}
}
\newcommand{\paraTodo}[4][]{%
	\cuantificador{#1}{\forall}{#2}{#3}{#4}
}

%listas

\newcommand{\TLista}[1]{\ensuremath{seq \langle #1\rangle}}
\newcommand{\lvacia}{\ensuremath{[\ ]}}
\newcommand{\lv}{\ensuremath{[\ ]}}
\newcommand{\longitud}[1]{\ensuremath{|#1|}}
\newcommand{\cons}[1]{\ensuremath{\mathsf{addFirst}}(#1)}
\newcommand{\indice}[1]{\ensuremath{\mathsf{indice}}(#1)}
\newcommand{\conc}[1]{\ensuremath{\mathsf{concat}}(#1)}
\newcommand{\cab}[1]{\ensuremath{\mathsf{head}}(#1)}
\newcommand{\cola}[1]{\ensuremath{\mathsf{tail}}(#1)}
\newcommand{\sub}[1]{\ensuremath{\mathsf{subseq}}(#1)}
\newcommand{\en}[1]{\ensuremath{\mathsf{en}}(#1)}
\newcommand{\cuenta}[2]{\mathsf{cuenta}\ensuremath{(#1, #2)}}
\newcommand{\suma}[1]{\mathsf{suma}(#1)}
\newcommand{\twodots}{\ensuremath{\mathrm{..}}}
\newcommand{\masmas}{\ensuremath{++}}
\newcommand{\matriz}[1]{\TLista{\TLista{#1}}}
\newcommand{\seqchar}{\TLista{\cha}}

\renewcommand{\lstlistingname}{Código}
\lstset{% general command to set parameter(s)
	language=Java,
	morekeywords={endif, endwhile, skip},
	basewidth={0.47em,0.40em},
	columns=fixed, fontadjust, resetmargins, xrightmargin=5pt, xleftmargin=15pt,
	flexiblecolumns=false, tabsize=4, breaklines, breakatwhitespace=false, extendedchars=true,
	numbers=left, numberstyle=\tiny, stepnumber=1, numbersep=9pt,
	frame=l, framesep=3pt,
	captionpos=b,
}

\usepackage{caratula} % Version modificada para usar las macros de algo1 de ~> https://github.com/bcardiff/dc-tex


\titulo{Descripci\'on del tp}
\subtitulo{Subtítulo del tp}

\fecha{\today}

\materia{Materia de la carrera}
\grupo{Grupo 42}

\integrante{Monte, Julian}{1266/23}{julianmonte03@gmail.com}
\integrante{Acosta Pozzoli, Lorenzo}{002/23}{email2@dominio.com}
\integrante{García Solá, Rodrigo Nahuel}{885/23}{rodri120gs@gmail.com}
\integrante{Martínez de Guereñu, Simon}{165/23}{email4@dominio.com}
% Pongan cuantos integrantes quieran

% Declaramos donde van a estar las figuras
% No es obligatorio, pero suele ser comodo
\graphicspath{{../static/}}

\begin{document}

\maketitle

\section{Especificación}
\subsection{Redistribucion de los frutos}

Calcula los recursos que obtiene cada uno de los individuos luego de que se redistribuyen
los recursos del fondo monetario común en partes iguales. El fondo monetario común se compone de la suma de recursos
iniciales aportados por todas las personas que cooperan. La salida es la lista de recursos que tendrá cada jugador


\begin{proc}{redistribucionDeLosFrutos}{\In recursos : \TLista{\float }, \In cooperan : \TLista{\bool}}{\TLista{\float }}
	\requiere{|recursos| > 0 \yLuego |recursos| = |cooperan| }
	\asegura{\paraTodo[unalinea]{i}{\ent} {0 \leq i < |cooperan|   \text{ }\yLuego\text{ }  cooperan[i] = True \implicaLuego res[i] = fondoComun(recursos, cooperan)/|recursos|} }
	\asegura{\paraTodo[unalinea]{i}{\ent} {0 \leq i < |cooperan|   \text{ }\yLuego\text{ }  cooperan[i] = False \implicaLuego res[i] = fondoComun(recursos,cooperan)/|recursos| + recursos[i]} }
	\aux{fondoComun}{\In recursos : \TLista{\float }, \In cooperan : \TLista{\bool}}{\float }{
	\sum\limits_{i=0}^{|recursos|-1} \begin{cases} recursos[i] & \text{si } cooperan[i] = \text{True} \\ 0 & \text{si } cooperan[i] = \text{False} \end{cases}
}

\end{proc}

\subsection{trayectoriaDeLosFrutosIndividualesALargoPlazo}

Actualiza (In/Out) la lista de trayectorias de los los recursos
de cada uno de los individuos. Inicialmente, cada una de las trayectorias (listas de recursos) contiene un único elemento
que representa los recursos iniciales del individuo. El procedimiento agrega a las trayectorias los recursos que los
individuos van obteniendo a medida que se van produciendo los resultados de los eventos en función de la lista de pagos
que le ofrece la naturaleza (o casa de apuestas) a cada uno de los individuos, las apuestas (o inversiones) que realizan
los individuos en cada paso temporal, y la lista de individuos que cooperan aportando al fondo monetario común.


\begin{proc}{trayectoriaDeLosFrutosIndividualesALargoPlazo}{\Inout trayectorias : \TLista{ \TLista{\float}}, \In cooperan : \TLista{\bool}, \In apuestas : \TLista{\TLista{\float}}, \In pagos : \TLista{\TLista{\float}}, \In eventos : \TLista{\TLista{\nat}}}{}
	\requiere{|pagos| = |eventos| = |apuestas| = |cooperan|}
	\requiere{\paraTodo[unalinea]{i}{\ent} {0 \leq i < |pagos| \implicaLuego \paraTodo[unalinea]{n}{\ent} {0 \leq n < |pagos[i]| \implicaLuego pagos[i][n] > 0}}}
	\requiere{\paraTodo[unalinea]{i}{\ent}{0 \leq i < |trayectoria| \implicaLuego |trayectoria[i]| > 0}}
	\asegura{\paraTodo[unalinea]{i}{\ent} {0 \leq i < |cooperan| \yLuego cooperan[i] = True \implicaLuego \paraTodo[unalinea]{n}{\ent} {1 \leq n < |trayectorias[i]| \implicaLuego trayectorias[i][n] = \newline \frac{\sum\limits_{a=0}^{|trayectorias|-1} \begin{cases} trayectorias[a][n-1] \times apuestas[a][eventos[a][n-1]] \times pagos[a][eventos[a][n-1]] & 		\text{si } cooperan[i] = \text{True} \\ 0 & \text{si } cooperan[i] = \text{False} \end{cases}}{|trayectorias|} }}\newline\newline}
	\asegura{\paraTodo[unalinea]{i}{\ent} {0 \leq i < |cooperan| \yLuego cooperan[i] = False \implicaLuego \paraTodo[unalinea]{n}{\ent} {1 \leq n < |trayectorias[i]| \implicaLuego trayectorias[i][n] = \newline  trayectorias[i][n-1] \times apuestas[i][eventos[i][n-1]] \times pagos[i][eventos[i][n-1]] + \newline \frac{\sum\limits_{a=0}^{|trayectorias|-1} \begin{cases} trayectorias[a][n-1] \times apuestas[a][eventos[a][n-1]] \times pagos[a][eventos[a][n-1]] & \text{si } cooperan[i] = \text{True} \\ 0 & \text{si } cooperan[i] = \text{False} \end{cases}}{|trayectorias|} }}}


\end{proc}

\subsection{trayectoriaExtrañaEscalera}

Esta función devuelve True sii en la trayectoria de un individuo existe un único punto
mayor a sus vecinos (llamado máximo local). Un elemento es máximo local si es mayor estricto que sus vecinos
inmediatos.



\begin{proc}{trayectoriaExtrañaEscalera}{\In trayectoria : \TLista{\float},}{\bool}
	\requiere{|trayectoria| > 0}
	\asegura{|trayectoria| = 1 \implicaLuego res = True}
	\asegura{(|trayectoria| = 2 \yLuego \paraTodo[unalinea]{i, n}{\ent}{0 \leq i , n< |trayectoria| \implicaLuego (i \neq n \implicaLuego trayectoria[i] \neq trayectoria[n])} \implicaLuego res = True}
	\asegura{(|trayectoria| > 2 \yLuego \existe[unalinea]{i}{\ent}{0 \leq i < |trayectoria| \implicaLuego (trayectoria[i-1] < trayectoria[i] \land trayectoria[i+1] < trayectoria[i]})) \implica res = True}


\end{proc}

\subsection{individuoDecideSiCooperarONo}

Un individuo actualiza su comportamiento cooperativo / no-cooperativo
(cooperan[individuo]) en función de los recursos iniciales, de quienes cooperan, de los pagos que se le ofrecen a cada
individuo, de las inversiones o apuestas de cada individuo, y del resultado los eventos que recibe cada individuo,
eligiendo el comportamiento que maximiza sus recursos individuales luego de que ocurren todos los eventos.



\begin{proc}{individuoDecideSiCooperarONo}{\In individuo : {\nat}, \In recursos: \TLista{\float}, \Inout cooperan: \TLista{\bool}, \In apuestas: \TLista{\TLista{\float}}, \In pagos: \TLista{\TLista{\float}}, \In eventos: \TLista{\TLista{\nat}}}{\bool}
	\requiere{}
	\asegura{}


\end{proc}

Y se pueden citar ecuaciones con \verb|\eqref{nombreDeEq}|: \eqref{eq:1}

Ejemplo de itemizado:

\begin{itemize}
	\item Item 1
	\item Item 2
	\item Item 3
\end{itemize}

Ejemplo de enumerado con menor distancia entre items:

\begin{enumerate} \setlength\itemsep{0cm}
	\item Item 1
	\item Item 2
	\item Item 3
\end{enumerate}

Podemos escribir mucho texto. Mucho texto. Mucho texto. Mucho texto. Mucho texto. Mucho texto. Mucho texto. Mucho texto. Mucho texto. Mucho texto. Mucho texto.

Otro párrafo. Otro párrafo. Otro párrafo. Otro párrafo. Otro párrafo. Otro párrafo. Otro párrafo. Otro párrafo. Otro párrafo. Otro párrafo. Otro párrafo. Otro párrafo. Otro párrafo.

\vspace{0.3cm}

Le agregamos una separación entre párrafos. Le agregamos una separación entre párrafos. Le agregamos una separación entre párrafos. Le agregamos una separación entre párrafos. Le agregamos una separación entre párrafos.

\vspace{0.3cm}

La tabla \ref{tab:ejemplo} es un ejemplo de cómo se hace una tabla.

\begin{table}[h!]
	\centering
	\begin{tabular}{||l c c r||} 
		\hline
		Col1 & Col2 & Col2 & Col3 \\ [0.5ex] 
		\hline\hline
		1 & 6 & 87837 & 787 \\ 
		2 & 7 & 78 & 5415 \\
		3 & 545 & 778 & 7507 \\
		4 & 545 & 18744 & 7560 \\
		5 & 88 & 788 & 6344 \\
		\hline
	\end{tabular}
	\caption{Ejemplo de tabla}
	\label{tab:ejemplo}
\end{table}


La figura \ref{fig:subfigs} es un ejemplo de cómo se agrega una imagen.

\begin{figure}[ht]
	\centering
	\includegraphics[width=0.6\textwidth]{logo_dc.jpg}
	\caption{Ejemplo de figura}
	\label{fig:ejemplo}
\end{figure}

\begin{figure}[ht!]
	\begin{subfigure}{0.5\textwidth}
		\includegraphics[width=0.9\linewidth]{LaTeX-project} 
		\caption{Logo de LaTeX}
		\label{fig:subfig1}
	\end{subfigure}
	\begin{subfigure}{0.5\textwidth}
		\includegraphics[width=0.7\linewidth]{TeX}
		\caption{Logo de TeX}
		\label{fig:subfig2}
	\end{subfigure}
	\caption{Ejemplo para poner dos figuras juntas. Y citarlas por separado a (\subref{fig:subfig1}) y (\subref{fig:subfig2}).}
	% OJO: el caption siempre va antes del label
	\label{fig:subfigs}
\end{figure}



% Para hacer que quede todo en una misma linea, se puede usar minipage
%\begin{minipage}[t]{\textwidth}
	\begin{lstlisting}[caption={Ejemplo de código (usando los estilos de la cátedra, ver las macros para más detalles)},label=code:for]
res := 0;
i := 0;
while (i < s.size()) do
	res := res + s[i];
	i := i + 1
endwhile
	\end{lstlisting}
%\end{minipage}

Si se pone un label al \verb|lstlisting|, se puede referenciar: Código \ref{code:for}.


\subsection{Macros de la cátedra para especificar}

\begin{proc}{nombre}{\In paramIn : \nat, \Inout paramInout : \TLista{\ent}}{tipoRes}
	%    \modifica{parametro1, parametro2,..}
	\requiere{expresionBooleana1}
	\asegura{expresionBooleana2}
	\aux{auxiliar1}{parametros}{tipoRes}{expresion}
	\pred{pred1}{parametros}{expresion} 
\end{proc}

\aux{auxiliarSuelto}{parametros}{tipoRes}{expresion}
% \paraTodo{variable}{tipo}{expresion}
% \existe{variable}{tipo}{expresion}
% Pueden tener [unalinea] para que no se divida en varias lineas
\pred{predSuelto}{parametros}{\paraTodo[unalinea]{variable}{tipo}{algo \implicaLuego expresion}}
\pred{predSuelto}{parametros}{\existe[unalinea]{variable}{tipo}{algo \yLuego expresion}}



\end{document}
